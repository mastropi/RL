% Template source: ICRA6_and_Risk_LatexTemplate.tex (ICRA template 2015, received from Montserrat Guillen)
% LaTeX documentation: https://www.latex-project.org/help/documentation/
% of which User's Guide: https://www.latex-project.org/help/documentation/usrguide.pdf
% Also check related links: https://www.latex-project.org/help/links/
% Comprehensive Tex Archive Network: https://www.ctan.org/

%***************************
% List of mathematical symbols: https://oeis.org/wiki/List_of_LaTeX_mathematical_symbols
%***************************

\documentclass[11pt,A4paper]{article}
%\documentclass[11pt,twoside,A4paper]{article}

\usepackage[papersize={21cm,29.7cm},left=3.5cm,top=3cm,right=3.5cm,bottom=2.5cm]{geometry}
\usepackage{latexsym,enumerate}
\usepackage{amsmath,amsthm,amsopn,amstext,amscd,amsfonts,amssymb}
\usepackage{graphics,graphicx}
\usepackage{setspace}
%\usepackage[spanish]{babel}
\usepackage[english]{babel}
\usepackage[T1]{fontenc}
\usepackage{uarial}
\renewcommand{\familydefault}{\sfdefault}
\usepackage{blindtext}
\usepackage{titlesec}
\singlespace

\author{Daniel Mastropietro}
\title{Efficient policy learning from rare states leading to Queue Blocking}
\date{\today}

\begin{document}
\maketitle

\section{Introduction}

Following the paper by Massaro et al. \cite{Massaro2019}, we propose a reinforcement learning algorithm that learns the optimal policy of an M/D/2/1 queue with the aim of minimizing the probability of blocking.

The focus of the designed algorithm is to learn the optimal policy in an efficient way, where efficiency is thought of in terms of learning from the state-action pairs that are most informative of rewards.

The algorithm leverages the knowledge we have in advance about rewards in the system. In fact, the reward received by the agent is always zero except at the state-action pairs that lead to blocking, in which it receives a negative reward.

The problem can be formally stated as follows: one single queue of capacity $K$ receives requests to process two types of jobs, each with a Poisson-like arrival rate $\lambda_i$ and each being served in deterministic time, assumed to be equal for all job types. The agent needs to define a policy that decides whether to accept or not a new job of a given class arriving to the queue.



\section{Markov Decision Process model}

We model the queue and the managing agent with a Markov Decision Process (MDP) having the following characteristics:

\medskip
States: Occupancy level of the queue $s \in \{0, ..., K\}$, i.e. there are $K+1$ possible states.

Note that the state only records the \emph{total} number of jobs in the queue. This limits policies to functions of the total queue occupation, disrespectful of their type.

\smallskip
Actions: $a \in \{0, 1\}$ (i.e. reject or accept a new job, respectively). Both actions are possible for all states except for the last one $s = K$ for which only $a = 0$ is possible.

\smallskip
Rewards: the system gives a large penalty when the queue blocks. For all other states, the reward is set to 0 for all actions.

\bigskip
The transition probabilities are known: when a new job is accepted, the state always increases by 1.


\section{Prediction problem}
For the prediction problem, we set the policy to be fixed and to always accept an incoming job.


\section{Control problem}


\newpage

\bibliography{RL}
\bibliographystyle{plain}




\iffalse

\medskip
It is assumed that:

\[S \sim \epsilon(\lambda)\]
\[T \sim \epsilon(1/\tau)\]

\subsection{Detection with noise}
The experiment is prone to several sources of errors. One of them is the following: from time to time the arrival of a new muon may be \textit{incorrectly classified} as a decay detection, thus causing an incorrect measurement of the muon's decay time.

\medskip
In order to model different sources of error, let us define the following random variable:

$U$: "Decay time of a muon in the presence of \textit{noise}"

where the \textit{noise} is generated by the incorrect detection of a muon decay.

\medskip
How is $U$ distributed?

\[ \\
U =
\left \{
  \begin{tabular}{ll}
  $T$ & \textrm{if it's a genuine decay detection} \\
  $S$ & \textrm{if the decay time measurement was triggered by the arrival of another muon} \\
  \end{tabular}
\right.
\]

\begin{align*}
F_U(u) 	= P(U \leq u) 	= & P(T \leq u /Y\!=\!0) \; (1-p) + P(S \leq u /Y\!=\!1) \; p \\
						= & (1 - e^{-u/\tau}) \; 1/(1 + \lambda \tau) I\{u \geq 0\} + 
							(1 - e^{-\lambda u}) \; \lambda \tau /(1 + \lambda \tau) I\{u \geq 0\} \\
						\approx & (1 - e^{-u/\tau}) \; (1 - \lambda \tau) I\{u \geq 0\} +
							 (1 - e^{-\lambda u}) \; \lambda \tau I\{u \geq 0\}
\end{align*}

\[
f_U(u) = F'_U(u) \approx \left[ \frac{(1 - \lambda \tau)}{\tau} e^{-u/\tau} \; + \lambda^2 \tau e^{-\lambda u} \; \right] I\{u \geq 0\}
\textrm{ when } \lambda \tau << 1.
\]

Note that as $\lambda \tau \to 0, f_U(u) \to \frac{e^{-u/\tau}}{\tau} \; I\{u \geq 0\} = f_T(u)$, that is $U$ tends to be distributed as $T$.

\subsection{Maximum Likelihood estimation of the parameters}
\[
l(\lambda, \tau / \underline{u}) = \sum_{i=1}^{n} \log \left[ \frac{(1 - \lambda \tau)}{\tau} e^{-u_i/\tau} \; + \lambda^2 \tau e^{-\lambda u_i} \; \right]
\]

or equivalently:

\[
l(\lambda, \tau / \underline{u}) = -\lambda \sum_{i=1}^{n} u_i + 
											\sum_{i=1}^{n} {\log \left[ \frac{(1 - \lambda \tau)}{\tau} e^{-\frac{(1 - \lambda \tau)}{\tau} u_i} \; + \lambda^2 \tau \; \right]}
\]

\fi

\end{document}
